\chapter{Approach: from photo to point}
\label {chap:approach}

This chapter focuses on defining the problem and presenting a formalised approach on how to solve it. Section \ref{sec:approach_modelling} ponders on idea of modelling pictures into points of a hyperplane. After that has been defined, sections \ref{sec:approach_general} and \ref{sec:approach_formal} present formalisations of the whole process, the first focusing on more general aspects, while the latter on every single detail.

\section{Problem modelling}
\label{sec:approach_modelling}

Recommender systems rely on recommending similar content based on some traits of the items. When recommending photographs, there is the problem of how will that be represented in a hyperplane. A naive solution relies on having individual pixel as a separate dimension. However, this solution is problematic for multiple reasons. High number of dimensions increase the complexity and two pictures may have a different amount of pixels. Obviously, relying on raw data for recommendations is not a good option.

When looking at pictures, people tend to identify objects and other people in order to give photos meaning. These are the most important aspects in pictures for us, humans, not endless, meaningless pixels. 

The question then becomes: how can photographs be modelled into points of a hyperplane in such a way that a recommender system which looks at its neighbours will find others with similar ones with similar objects? 

\section{General view}
\label{sec:approach_general}

To answer the question posed in section \ref{sec:approach_modelling}, it must be a solution which makes an emphasis on object detection. A first and crucial step is performing object detection on the input images.

However, object detection models which use neural networks are only capable of giving an ordered list of objects, sorted by confidence. Some models are capable of detecting hundreds or thousands of objects, so having a dimension with the frequency of each word will send the solution back to the pixel problem. 

In order to further reduce dimensions, the object names can be reduced by being associated with certain environments. And there are models which do such that to some degree. Word2vec transforms a word into a multidimensional vector \cite{word2vec}, which can then have certain operations performed on it in comparison with other words, such as similarity. This can prove as a good solution to reducing dimensions.

Using word2vec similarity operator as a mapping function can prove useful in obtaining a general membership of a picture to a certain theme. The only problem standing is selecting themes and choosing the right words to describe them.

To summarize: the photo to point transformation process starts by processing an input image, identifying its objects, then performing a weighted average of mapping of each word to a certain theme (where the theme is a word and the mapping function is the similarity function from the word2vec model). Finally, by having $d$ themes, each photograph is mapped into a $d$-dimensional hyperplane (figure \ref{fig:photo2point}).

\begin{figure}[b!]
\centering
\includegraphics[width=0.9\textwidth]{photo2point}
\caption{A summary of the photo to point process}
\label{fig:photo2point}
\end{figure}

\section{Solution formalisation}
\label{sec:approach_formal}

Let $x \in X$ be a photograph from the dataset $X$ and $d$ be the number of dimensions of the hyperplane. In order to map the input image $x$ to a hyperplane coordinate, a function $map$ should be defined as follows:
$$ map : X \mapsto \mathbf{R}^{d} $$

Now, as discussed in section \ref{sec:approach_general}, the mapping procedure is composed of 2 smaller sub-procedures: the first one maps the input image into a set of words (section \ref{subsec:approach_formal_obj}) while the second one maps the set of words into hyperdimensional points in the plane using a set of keywords (section \ref{subsec:approach_formal_word2vec}). So, the maping function  should reflect this:

$$ map : X, \bigtimes_{i=1}^{d}{vocabulary} \mapsto $$ $$ \mapsto \{(word, conf) | word \in vocabulary, conf \in [0, 100]\}, vocabulary \mapsto \mathbf{R}^{d} ,$$
and has the form:
$$ map(x, keywords) = det2point(obj2det(x), keywords) $$

\subsection{Object detection}
\label{subsec:approach_formal_obj}

A object detection model is able to identify objects in a certain image with a certain grade of confidence, typically in the interval $[0, 100]$. As mentioned in section \ref{sec:approach_formal}, in the definition of the $map$ function, the object detection function should have the following definition (the keywords are not used, so the second input variable can be safely excluded):
$$ img2obj : X \mapsto \{(word, conf) | word \in vocabulary, conf \in [0, 100]\} $$

As to how the $img2obj$ function exactly works, that can be considered a blackbox on paper, because in a practical situation it can be any neural network which depends on multiple, trainable weights, thus making it impossible to be described on paper.

\subsection{Word2Vec}
\label{subsec:approach_formal_word2vec}

The object detection function outputs a set of words, each with a certain grade of confidence. Using these and a tuple $keywords \in \bigtimes_{i=1}^{d}{vocabulary}$ as inputs for the function $det2point$, the second part of hyperplane mapping is performed, which will be presented in detail in this section.

Recall from chapter \ref{chap:words} that word embeddings have the role of mapping words into coordinate in plane. Which means, there exists a function $word2vec : vocabulary \mapsto \mathbf{R}^{d_{word2vec}} $, where $d_{word2vec} \in \mathbf{N^{*}}$ is the dimension of the model's hyperplane.

As mentioned in section \ref{sec:approach_general}, a similarity function between two word2vec mappings is used for the final mapping. The most commonly used similarity function is the cosine similarity function $cos(\theta) = \frac{a b}{|| a || || b ||} $. 

Let $w_1, w_2 \in vocabulary$, where $w_1$ is a word from the object detection set and $w_2 \in keywords$. Then, the theme mapping function is defined as:
$$ map_{word}(w_1, w_2) =  \frac{word2vec(w_1), word2vec(w_2)}{|| word2vec(w_1) || \cdot || word2vec(w_2) ||} $$

But this is just one pair of words. At this step, the inputs are the objects detected in the image with their respective grade of confidence and a set of keywords. To fully map the input to each themed dimension from the keywords, the similarity function is applied for all pairs. Since the object detection provides a grade of confidence of each detection, that can be used to prevent the model in becoming too confident on a mapping based on inaccurate data. Thus, the theme mapping applied on the out from the object detection process is $ det2point(det, keyword) = (y_1, y_2, ..., y_d) $, where 
$$ y_i = avg_{w}(map_{word} (det_{words}, keyword_i) , det_{conf} ) ,$$

where $avg_{w}(x, w) = \frac{x_1 w_1 + x_2 w_2 + ... + x_n w_n}{\sum_{i=1}^{n}{w_i}}$ is the weighted average of the vector $x \in \mathbf{R}^n$ with the weights $w \in \mathbf{R}^n$ and $map_{word} (v, w)$ is the vector resulted from applying $map_{word}$ on each item $v_i$ in the vector on the word $w$.

\subsection{Cluster querying}
\label{subsec:approach_formal_cluster}

When going through all the steps in sections \ref{subsec:approach_formal_obj} and \ref{subsec:approach_formal_word2vec}, the mapping result is associated to the input photo, but in order for recommendations to work, the operations, which will be performed on the output hyperdimensional space, will result in a vector which is also part of said space, need to be mapped back. Mathematically speaking, assume that $\forall x_1, x_2 \in X \implies map(x_1, keywords) \neq map(x_2, keywords)$. Which allows the definition of the inverse function $map^{-1} : \{ map(x) | \forall x \in X \} \mapsto X$. Since the keywords are already known, their output value can be ignored safely.

Then, a recommendation $rec: X \mapsto X$ for an input image $x$ is defined as:
$$rec(x, K) = map^{-1}(nn(map(x, keywords), K)),$$
where $nn(v, K)$ returns the closest $K$ vectors to input vector $v$ using the euclidean distance metric:
$$ dist(a, b) = \sqrt[d]{\sum_{i=1}^{d}{(a_i - b_i)^2}} $$
