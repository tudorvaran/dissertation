\chapter{Conclusion and future work}
\label{chap:conclusion}

This paper investigated how a recommender system for pictures based on the objects in them can be modelled. Most current solutions rely on manual labelling (or tagging) with words, which would then act just as search terms. This solution approached the problem from a different perspective, by attempting to map various-sized images into a hyperplane, using object detection and word similarity as intermediary steps. 

Since the model is an unsupervised model, another recommender with ground truth data was used for validation. As described in section \ref{sec:results_scaler} the model obtained a precision at 1 (P@1) of $0.137$ (compared to $0.003$ if selecting random entries instead).

This project already has real-world applicability in the form of a website, but it relies on defining a set of keywords to act as dimensional labels. This is a topic that can be always be improved through continuous research to improve quality based on the selected set.

Besides improving the precision of the model, another feature that can be added on top is face detection. It can be used to make distinctions between faces and recommending pictures with the same people in them. However, this subject adds extra complexity to the project and longer investigations are necessary on ways to integrate it.